\documentclass[12pt,twoside]{article}
\linespread{1.5} 
\usepackage[T1]{fontenc}
\usepackage{textalpha}

\usepackage{textgreek}
\usepackage[english, greek]{babel}
\usepackage[numbers]{natbib}
\usepackage{url}
\usepackage[utf8x]{inputenc}
\usepackage{amsmath, amssymb, amsfonts}
\usepackage{graphicx}
\graphicspath{{images/}}
\usepackage{rotating}
\usepackage{fancyhdr}
\usepackage{booktabs}
\usepackage{tabularx}
\usepackage{comment} 
\usepackage{lettrine}
\usepackage{type1cm}
\usepackage{vmargin}
\setmarginsrb{3cm}{1.5cm}{3cm}{2.5cm}{1cm}{1.5cm}{1cm}{1.5cm}
\usepackage{bm}
\usepackage{csquotes}
\renewcommand*\descriptionlabel[1]{\hspace\leftmargin$#1$}
\usepackage{enumitem} 
\usepackage{commath}
\usepackage{listings}
\usepackage{breakcites}
\usepackage[hidelinks]{hyperref}
\usepackage{footnote}
\usepackage[nottoc]{tocbibind}
\usepackage[caption=false, font=footnotesize]{subfig}
\usepackage[export]{adjustbox}

\usepackage{algorithm}
\usepackage{algorithmic}

\renewcommand{\algorithmicrequire}{\textbf{Input:}}

\renewcommand{\algorithmicensure}{\textbf{Output:}}
\setcitestyle{square}

\newcommand{\en}{\selectlanguage{english}}
\newcommand{\gr}{\selectlanguage{greek}}
\newcommand\blfootnote[1]{%
  \begingroup
  \renewcommand\thefootnote{}\footnote{#1}%
  \addtocounter{footnote}{-1}%
  \endgroup
}

\gr
\title{\gr Ανάπτυξη συστήματος αναγνώρισης μεροληψίας σε μεθόδους μηχανικής μάθησης}

\pagestyle{fancy}
\fancyhf{}
\fancyhead[LE,RO]{\itshape\nouppercase{\rightmark}}
\cfoot{\thepage}

\begin{document}
\raggedbottom

\begin{titlepage}
\en
    \centering
    \includegraphics[scale = 0.75]{logo.png}\\[0.0 cm]  % University Logo
    \textsc{\gr \Huge Πολυτεχνική Σχολή \\ Τμημα μηχανικων η/υ και πληροφορικης}\\[0.5 cm]  % University Name
    \rule{\linewidth}{0.2 mm} \\[0.4 cm]
    { \Large \bfseries Ανάπτυξη συστήματος αναγνώρισης μεροληψίας σε μεθόδους μηχανικής μάθησης}\\
    \rule{\linewidth}{0.2 mm} \\[0.4 cm]
    \gr \textbf{\Large{ΔΙΠΛΩΜΑΤΙΚΗ ΕΡΓΑΣΙΑ}\\}
    \large {του Ορέστη Ι. Τσαγκέτα }\\[1.0cm]
    \begin{minipage}{0.4\textwidth}
        \begin{flushleft} \normalsize
            \textbf{\emph{\gr Επιβλέπων:}\\}
            \gr Χρήστος Μακρής\\
            \gr Αναπληρωτής Καθηγητής\\
            \gr Τμήμα Μηχανικών Η/Υ και Πληροφορικής\\
            \gr Πανεπιστήμιο Πατρών\\
        \end{flushleft}
    \end{minipage}~
    \begin{minipage}{0.4\textwidth}
        \begin{flushright} \normalsize
            \textbf{\emph{\gr Συνεπιβλέπον:} \\}
            \gr Ιωάννης Κανελλόπουλος\\
	 \vspace{4\baselineskip} % Adjust the space here to match the other minipage
        \end{flushright}
    \end{minipage}\\[0.5 cm]
    \gr \small {\textbf{Πάτρα, Σεπτέμβριος 2024 }}
\end{titlepage}


\pagenumbering{arabic}
\newpage
\vspace*{\fill}
\hspace{-7.5mm}
\textbf{\en Copyright © \gr Ορέστης Ι. Τσαγκέτας, 2024\\
Με επιφύλαξη παντός δικαιώματος. \en All rights reserved.\gr\\}
\hspace{-7.5mm}\\
\textbf{Απαγορεύεται}  η  αντιγραφή, η αποθήκευση  και η διανομή  της  παρούσας  εργασίας,  εξ’ ολοκλήρου ή τμήματος αυτής, για εμπορικό σκοπό.
Επιτρέπεται η ανατύπωση, η αποθήκευση και η διανομή  για  σκοπό  μη-κερδοσκοπικό,  εκπαίδευσης  ή  ερευνητικής  φύσης,  υπό  την 
προϋπόθεση  να  αναφέρεται  η  πηγή  προέλευσης  και  να διατηρείται το  παρόν  μήνυμα. 
Ερωτήματα  που  αφορούν  τη  χρήση  της  εργασίας  για  κερδοσκοπικό  σκοπό  πρέπει  να 
απευθύνονται προς τον συγγραφέα.\\
\hspace{-7.5mm}\\
Οι  απόψεις  και  τα  συμπεράσματα  που  περιέχονται  σε  αυτό  το  έγγραφο  εκφράζουν  τον 
συγγραφέα  και  δεν  πρέπει  να  ερμηνευθεί  ότι  αντιπροσωπεύουν  τις  επίσημες  θέσεις  του 
Πανεπιστημίου Πατρών.

\thispagestyle{empty}
\newpage

%ΠΕΡΙΛΗΨΗ
\thispagestyle{plain}
\paragraph{\Large{\gr Περίληψη}}
\
\\
\vspace{-0.7cm}
\gr
\lettrine[loversize=0.03]{\en H} \gr παρούσα διπλωματική εργασία ασχολείται με τη σχεδίαση και ανάπτυξη ενός εργαλείου που στοχεύει στον εντοπισμό και τη μείωση της μεροληψίας σε μεθόδους μηχανικής μάθησης. Το εργαλείο, το οποίο είναι μια διαδικτυακή εφαρμογή, αναπτύχθηκε σε \en Python \gr χρησιμοποιώντας το \en Flask Framework \gr και την εργαλειοθήκη \en Aif360 \gr της \en IBM \gr. Οι χρήστες της εφαρμογής καλούνται να ανεβάσουν ένα \en dataset \gr με το οποίο εκπαιδεύουν ένα από τα διαθέσιμα μοντέλα μηχανικής μάθησης. Ο έλεγχος της μεροληψίας πραγματοποιείται βάσει του χαρακτηριστικού που επιλέγει ο χρήστης να ελεγχθεί, και αν επιβεβαιωθούν οι υποψίες του, να μειωθεί. Αυτή η διαδικασία διενεργείται μέσω της εφαρμογής, χρησιμοποιώντας διάφορες μετρικές μεροληψίας. Η μείωση της μεροληψίας επιτυγχάνεται με τη χρήση συγκεκριμένων αλγορίθμων, τους οποίους ο χρήστης μπορεί να επιλέξει ανάλογα με τις ανάγκες του. Για τη σωστή επιλογή των μετρικών και των αλγορίθμων, η εφαρμογή παρέχει καθοδήγηση στον χρήστη, λαμβάνοντας υπόψη τους περιορισμούς που προκύπτουν από τα χαρακτηριστικά των δεδομένων. Ο βασικός στόχος αυτής της διπλωματικής εργασίας είναι η εκπαίδευση και εξοικείωση των χρηστών που δεν διαθέτουν προγραμματιστικές γνώσεις ή βαθιά κατανόηση της μηχανικής μάθησης με την έννοια της δικαιοσύνης στους αλγορίθμους μηχανικής μάθησης. Επιπλέον, η εφαρμογή θα ελεγχθεί ώστε τα αποτελέσματά της να συμμορφώνονται με τον νόμο \en Local Law 144 of 2021 \gr, που επιβάλλεται από το \en NYC Department of Consumer and Worker Protection (DCWP) \gr. Ο νόμος αυτός απαιτεί διαφάνεια και δίκαιες πρακτικές στις αποφάσεις που λαμβάνονται μέσω αυτοματοποιημένων συστημάτων λήψης αποφάσεων, διασφαλίζοντας ότι δεν υπάρχει μεροληψία κατά συγκεκριμένων ομάδων πληθυσμού.

\paragraph{\large{\gr Λέξεις Κλειδιά: Αλγοριθμική Δικαιοσύνη, Μετρικές Δικαιοσύνης, Αλγόριθμοι Μείωσης Μεροληψίας, Μηχανική Μάθηση , \en Python\gr, \en Aif360\gr, \en Flask\gr, \en Local Law 144 of 2021\gr}}
%ABSTRACT
\newpage
\thispagestyle{plain}
\paragraph{\Large{\en Abstract}}
\
\\
\vspace{-0.7cm}
\en
\lettrine [loversize = 0.03] {\en T} {\ he} current Diploma Thesis focuses on the design and development of a tool aimed at detecting and reducing bias in machine learning methods. The tool, which is a web application, was developed in Python using the Flask Framework and the Aif360 toolkit from IBM. Users of the application are required to upload a dataset with which they train one of the available machine learning models. Bias detection is conducted based on the characteristic selected by the user to be checked, and if their suspicions are confirmed, to be reduced. This process is carried out through the application using various bias metrics. Bias reduction is achieved using specific algorithms that the user can choose based on their needs. For the correct selection of metrics and algorithms, the application provides guidance to the user, taking into account the constraints arising from the characteristics of the data. The primary goal of this thesis is to educate and familiarize users who do not have programming knowledge or a deep understanding of machine learning with the concept of fairness in machine learning algorithms. Additionally, the application will be tested to ensure its results comply with Local Law 144 of 2021, enforced by the NYC Department of Consumer and Worker Protection (DCWP). This law requires transparency and fair practices in decisions made through automated decision systems, ensuring that there is no bias against specific population groups.

\paragraph{\large{\en Keywords: Algorithmic Justice, Justice Metrics, Bias Reduction Algorithms, Machine Learning, Python, Aif360, Flask, Local Law 144 of 2021}}

\newpage\null
\thispagestyle{plain}
\vspace*{\fill}
\begin{center}
\begin{minipage}{.6\textwidth}
\centering \textbf{\en\textit{ “Being good is easy, what is difficult is being just.”}
\rightline{{\rm ---  Victor Hugo}}}%\gr Σκοπίμως κενή σελίδα}
\end{minipage}
\end{center}
\vfill

%ΕΥΧΑΡΙΣΤΙΕΣ
\newpage
\thispagestyle{plain}
\paragraph{\Large{\gr Ευχαριστίες}}
\paragraph{}
\gr Θα ήθελα να ευχαριστήσω κ. Γ. Κανελλόπουλο και τον καθηγητή  κ. Χρήστο Μακρή  και  για την επίβλεψη  αλλά και για τη συμβολή τους στην εκπόνηση αυτής της διπλωματικής εργασίας.
\paragraph{}
Τέλος θα ήθελα να ευχαριστήσω τους γονείς μου για τη υποστήριξη και την ηθική συμπαράσταση που μου προσέφεραν όλα αυτά τα χρόνια.

\hspace{8cm} \textbf{\textit{Πάτρα, 1 Σεπτεμβρίου 2024}}

\newpage
\thispagestyle{empty}
\null

\gr
\newpage
\thispagestyle{plain}
\tableofcontents

\gr
\newpage
\thispagestyle{plain}
\listoffigures

\newpage
\thispagestyle{plain}
\listoftables

\newpage\null
\thispagestyle{empty}

\pagebreak

%%%%%%%%%%%%%%%%%%%%%%%%%%%%%%%%%%%%%%%%%%%%%%%%%%%%%%%%%%%%%%%%%%%%%%%%%%%%%%%%%%%%%%%%%

%ΕΙΣΑΓΩΓΗ
\newpage
\thispagestyle{plain}
\null
\vspace{2cm}
\hspace{-6.5mm}{\textbf{\Huge
 \gr Κεφάλαιο 1: Εισαγωγή}}
\vspace{-4mm}
\section{\gr Εισαγωγή}
\gr

\paragraph{}\gr
Η ευρεία υιοθέτηση της μηχανικής μάθησης (ML) σε διάφορους τομείς, από εγκρίσεις δανείων και συστήματα αναγνώρισης προσώπου μέχρι προβλέψεις στην ποινική δικαιοσύνη, έχει φέρει σημαντικά οφέλη στην κοινωνία. Ωστόσο, υπάρχει μια αυξανόμενη ανησυχία σχετικά με τη δυνατότητα εμφάνισης προκαταλήψεων σε αυτούς τους ισχυρούς αλγόριθμους \cite{Carvalho2019, Gebru2019}. Οι προκαταλήψεις στα μοντέλα ML μπορούν να οδηγήσουν σε άδικα και διακριτικά αποτελέσματα, ιδιαίτερα όταν εμπλέκονται ευαίσθητα δεδομένα, πιθανώς διαιωνίζοντας τις υπάρχουσες κοινωνικές ανισότητες \cite{Kleinberg2018}.

Αυτή η διπλωματική εργασία ασχολείται με το κρίσιμο ζήτημα της ανίχνευσης και μείωσης των προκαταλήψεων στις μεθόδους ML. Παρουσιάζουμε το σχεδιασμό και την ανάπτυξη μιας φιλικής προς τον χρήστη διαδικτυακής εφαρμογής που δίνει τη δυνατότητα σε άτομα, ακόμη και χωρίς εκτεταμένη γνώση προγραμματισμού, να εντοπίζουν και να μειώνουν τις πιθανές προκαταλήψεις στα μοντέλα μηχανικής μάθησης τους.

Αυτή η εργασία συμβάλλει σημαντικά στον τομέα της Αλγοριθμικής Δικαιοσύνης \cite{Selbst2019} παρέχοντας ένα πρακτικό εργαλείο που ενισχύει τη διαφάνεια και προάγει τις ανησυχίες δικαιοσύνης καθ' όλη τη διάρκεια ανάπτυξης των μοντέλων ML. Η εφαρμογή αξιοποιεί το ισχυρό εργαλείο Aif360 από την IBM \cite{IBM2023} για την ανάλυση των δεδομένων που ανεβάζουν οι χρήστες και την ανίχνευση πιθανών προκαταλήψεων βάσει καθορισμένων από τον χρήστη χαρακτηριστικών, όπως η φυλή, το φύλο ή η ηλικία. Αυτό επιτρέπει στους χρήστες να εντοπίζουν περιοχές όπου τα μοντέλα τους μπορεί να παρουσιάζουν άδικες προτιμήσεις προς συγκεκριμένες δημογραφικές ομάδες.

Επιπλέον, η εφαρμογή προχωρά πέρα από την απλή ανίχνευση προκαταλήψεων, προτείνοντας κατάλληλες τεχνικές μείωσης προκαταλήψεων. Η εφαρμογή συνιστά κατάλληλους αλγόριθμους για τη μείωση του ανιχνευόμενου τύπου προκατάληψης, λαμβάνοντας υπόψη τα συγκεκριμένα χαρακτηριστικά των δεδομένων και τις πιθανές περιορισμούς. Αυτό δίνει τη δυνατότητα στους χρήστες να αντιμετωπίζουν ενεργά τις ανησυχίες δικαιοσύνης και να διασφαλίζουν ότι τα μοντέλα ML τους λειτουργούν με υπεύθυνο και ηθικό τρόπο.

Η εφαρμογή προωθεί τη συμμόρφωση με κανονισμούς όπως ο Τοπικός Νόμος 144 του 2021 που επιβάλλεται από το Τμήμα Προστασίας Καταναλωτών και Εργαζομένων της Νέας Υόρκης (DCWP) \cite{DCWP2021}. Αυτός ο νόμος απαιτεί διαφάνεια και δικαιοσύνη στα αυτοματοποιημένα συστήματα αποφάσεων, ευθυγραμμιζόμενος απόλυτα με τον στόχο μας για την προώθηση υπεύθυνων και ηθικών πρακτικών AI. Με την ενεργή μείωση των προκαταλήψεων και τη διασφάλιση της διαφάνειας στη διαδικασία ανάπτυξης μοντέλων, η εφαρμογή δίνει τη δυνατότητα στους χρήστες να συμμορφώνονται με τέτοιους κανονισμούς χωρίς να διακυβεύεται η λειτουργικότητα των μοντέλων ML τους.

\subsection{\gr Πρόβλημα}
\paragraph{}\gr Υπάρχουν πολλές αξιοσημείωτες περιπτώσεις που υπογραμμίζουν τη σημασία της δικαιοσύνης στα \en AI \gr συστήματα. Ένα χαρακτηριστικό παράδειγμα είναι το αυτοματοποιημένο εργαλείο πρόσληψης της \en Amazon \gr \cite{buddautomated} \cite{lewis2018will}. Ξεκίνησε το \en 2014, \gr αυτό το εργαλείο χρησιμοποίησε ορισμένους αλγορίθμους για την αξιολόγηση βιογραφικών και την βαθμολόγηση υποψηφίων. Ωστόσο, μέχρι το \en 2015, \gr διαπιστώθηκε ότι το σύστημα πρόσληψης δεν βαθμολογούσε δίκαια τους υποψηφίους, καθώς ευνοούσε τους άντρες υποψηφίους έναντι των γυναικών. Αυτή η μεροληψία προέκυψε επειδή το εργαλείο είχε εκπαιδευτεί με βιογραφικά που είχαν υποβληθεί στην \en Amazon \gr κατά τη διάρκεια μιας δεκαετίας, τα περισσότερα από τα οποία προέρχονταν από άντρες \cite{buddautomated} \cite{lewis2018will}.

\gr Ένα άλλο χαρακτηριστικό παράδειγμα είναι το λογισμικό \en COMPAS \gr (Correctional Offender Management Profiling for Alternative Sanctions) \cite{lewandowski2021machine} \cite{goel2021accuracy} \gr που χρησιμοποιήθηκε από την κυβέρνηση των \en Η.Π.Α., \gr το οποίο υπολόγιζε βάσει των δεδομένων των κατηγορουμένων ένα σκορ (1 έως 10). Αυτά τα σκορ βοηθούσαν τους δικαστές να αποφασίσουν για ποινές, αναστολές και άλλες δικαστικές αποφάσεις. Ωστόσο, μελέτες αποκάλυψαν ότι ο αλγόριθμος είχε μεγαλύτερη πιθανότητα να προβλέψει λανθασμένα ότι οι μαύροι κατηγορούμενοι θα επαναλάμβαναν το έγκλημα σε σύγκριση με τους λευκούς κατηγορούμενους, οδηγώντας σε δυσανάλογες επιπτώσεις στις αποφάσεις για την ποινή και την αναστολή \cite{goel2021accuracy} \cite{lewandowski2021machine}.

\gr Αυτά τα παραδείγματα υπογραμμίζουν την κρίσιμη ανάγκη αντιμετώπισης των μεροληψιών στα \en AI \gr συστήματα για να διασφαλιστεί η δικαιοσύνη και η ισότητα. Η έρευνα συνεχίζει να εξερευνά μεθόδους για τον μετριασμό αυτών των μεροληψιών, όπως η ανάπτυξη πιο διαφανών αλγορίθμων και η ενσωμάτωση μετρήσεων δικαιοσύνης στον σχεδιασμό και την αξιολόγηση των \en AI \gr συστημάτων \cite{mitchell2021algorithmic} \cite{mehrabi2021survey}. Μία προσέγγιση είναι η χρήση τεχνικών μηχανικής μάθησης που είναι ευαίσθητες στη δικαιοσύνη και προσαρμόζουν τη διαδικασία μάθησης για να ελαχιστοποιήσουν τη μεροληψία. Μια άλλη μέθοδος είναι η διενέργεια ελέγχων μεροληψίας για τον εντοπισμό και τη διόρθωση των μεροληψιών πριν από την ανάπτυξη των \en AI \gr συστημάτων σε κρίσιμες εφαρμογές.

\gr Για παράδειγμα, η ενσωμάτωση περιορισμών δικαιοσύνης στη διαδικασία εκπαίδευσης μπορεί να βοηθήσει στη διασφάλιση ότι τα παραγόμενα μοντέλα δεν επηρεάζουν δυσανάλογα καμία συγκεκριμένη ομάδα \cite{dwork2012fairness}. Επιπλέον, η χρήση τεχνικών εξηγήσιμης \en AI \gr (\en XAI \gr) μπορεί να προσφέρει πληροφορίες για το πώς λαμβάνονται οι αποφάσεις από τα \en AI \gr συστήματα, καθιστώντας ευκολότερο τον εντοπισμό και τη διόρθωση της μεροληπτικής συμπεριφοράς.

\gr Οι πρόσφατες εξελίξεις στη μείωση της μεροληψίας περιλαμβάνουν την ανάπτυξη τεχνικών αντίπαλης αποπροκατάληψης (\en adversarial debiasing \gr), οι οποίες περιλαμβάνουν την εκπαίδευση των \en AI \gr μοντέλων με τρόπο που οι αντίπαλοι προσπαθούν να εισαγάγουν μεροληψία και το κύριο μοντέλο μαθαίνει να την εξουδετερώνει \cite{zhang2018mitigating}. Αυτή η μέθοδος έχει δείξει υποσχέσεις για τη μείωση της μεροληψίας σε διάφορες εφαρμογές, από τις προσλήψεις μέχρι τη δικαιοσύνη.

\gr Επιπλέον, υπάρχει αυξανόμενη έμφαση στη σημασία των διεπιστημονικών προσεγγίσεων, συνδυάζοντας γνώσεις από την πληροφορική, το δίκαιο, την ηθική και τις κοινωνικές επιστήμες για την αντιμετώπιση της μεροληψίας με ολοκληρωμένο τρόπο. Οι συνεργατικές προσπάθειες και οι περιεκτικές ερευνητικές πρακτικές είναι απαραίτητες για την ανάπτυξη ανθεκτικών λύσεων που εξασφαλίζουν ότι οι τεχνολογίες \en AI \gr ωφελούν δίκαια όλα τα τμήματα της κοινωνίας.

\gr Συνολικά, η αντιμετώπιση της μεροληψίας στα \en AI \gr είναι μια σύνθετη και διαρκής πρόκληση που απαιτεί συνεχή προσπάθεια και συνεργασία μεταξύ των επιστημών. Με τον συνδυασμό τεχνικών λύσεων με πολιτικές και κανονιστικά μέτρα, είναι δυνατόν να δημιουργηθούν \en AI \gr συστήματα που δεν είναι μόνο αποτελεσματικά αλλά και δίκαια και σωστά.


\subsection{\gr Δομή Διπλωματικής Εργασίας}
\paragraph{}\gr Στα πλαίσια της διπλωματικής εργασίας μελετήθηκαν διάφορες διεθνείς δημοσιεύσεις που αφορούν την έννοια της αλγοριθμικής δικαιοσύνης, τις μετρικές εκτίμησης της αλγοριθμικής μεροληψίας και τους αλγορίθμους μείωσης της. Η ιστοσελίδα κατασκευάστηκε με χρήση του \en Flask \gr \cite{grinberg2018flask}, ενώ οι λειτουργίες της που αφορούν τη μέτρηση και τη μείωση της αλγοριθμικής μεροληψίας υλοποιήθηκαν με τη βιβλιοθήκη \en AI Fairness 360 \gr της \en IBM \cite{bellamy2018ai}.

\paragraph{}\gr Στο κεφάλαιο 2 γίνεται βιβλιογραφική ανασκόπηση σχετικά με τη μηχανική μάθηση. Αρχικά, γίνεται αναλυτική παρουσίαση της έννοιας της αλγοριθμικής μεροληψίας και των σταδίων στα οποία μπορεί να εμφανιστεί. Στη συνέχεια, περιγράφεται η διαδικασία δημιουργίας ενός μοντέλου μηχανικής μάθησης με έμφαση στα \en binary \gr μοντέλα. Εξετάζονται τα κριτήρια αξιολόγησης των μοντέλων και παρουσιάζονται οι μετρικές και οι αλγόριθμοι μείωσης της μεροληψίας που συμπεριλήφθηκαν στην παρούσα εργασία. Επιπλέον, γίνεται μελέτη των νομικών και ηθικών ζητημάτων, όπως ο νόμος \en Local Law 144 of 2021 \gr.

\paragraph{}\gr Στο κεφάλαιο 3 παρουσιάζεται αναλυτικά η αρχιτεκτονική του συστήματος, ο σχεδιασμός και οι τεχνολογίες στις οποίες βασίζεται. Αναλύεται το σενάριο χρήσης, τα λογικά διαγράμματα και καταλήγει με την ανάπτυξη του κώδικα.

\paragraph{}\gr Στο κεφάλαιο 4 περιγράφεται η διαδικασία αξιολόγησης της εφαρμογής και τα αποτελέσματα που προέκυψαν από αυτή.

\paragraph{}\gr Στο κεφάλαιο 5 παρατίθενται τα συμπεράσματα από την αξιολόγηση και τα αποτελέσματα της εργασίας. Τέλος, σκιαγραφούνται οι μελλοντικές ερευνητικές κατευθύνσεις και τα ζητήματα που προκύπτουν από την εργασία, καθώς και οι περιορισμοί που εντοπίστηκαν κατά την υλοποίηση και την αξιολόγηση. Συγκεκριμένα, επισημαίνονται τα προβλήματα και οι προκλήσεις που συνδέονται με την εξασφάλιση της αλγοριθμικής δικαιοσύνης σε διαφορετικά πλαίσια εφαρμογής και προτείνονται λύσεις και κατευθύνσεις για περαιτέρω έρευνα. Οι περιορισμοί που εντοπίστηκαν περιλαμβάνουν την ανάγκη για μεγαλύτερα και πιο ποικίλα δεδομένα εκπαίδευσης, την αυξημένη πολυπλοκότητα των αλγορίθμων μείωσης της μεροληψίας, καθώς και την ανάγκη για συνεχή ενημέρωση και προσαρμογή στις τρέχουσες νομικές και ηθικές απαιτήσεις.

\subsection{\gr Συνεισφορά}
\paragraph{}\gr Η παρούσα διπλωματική εργασία εστιάζει στην ανάπτυξη ενός εργαλείου αξιολόγησης της αλγοριθμικής δικαιοσύνης, το οποίο ευθυγραμμίζεται πλήρως με τις προδιαγραφές του Νόμου 144 του 2021 (\en "Local Law 144 of 2021" \gr) της Νέας Υόρκης. Στόχος μας είναι να καταστήσουμε το εργαλείο εύχρηστο και προσβάσιμο σε χρήστες με ή χωρίς εξειδικευμένες γνώσεις, ώστε να μπορούν να αξιολογούν την δίκαιη λειτουργία των μοντέλων τεχνητής νοημοσύνης που υλοποιούν, να εντοπίζουν τυχόν προκαταλήψεις και να υιοθετούν στρατηγικές για την μείωση ή την εξάλειψή τους.

\paragraph{}\gr Πέρα από την τήρηση του νομικού πλαισίου, το εργαλείο μας φιλοδοξεί να προσφέρει ουσιαστική αξία στον πραγματικό κόσμο, συμβάλλοντας στην υιοθέτηση ηθικών και δίκαιων εφαρμογών της τεχνητής νοημοσύνης σε διάφορους τομείς.



\newpage
\gr
\en
\thispagestyle{plain}
\renewcommand{\bibname}{\gr Βιβλιογραφία}
\bibliographystyle{unsrt}%unsrt}
\bibliography{references}
\end{document}
